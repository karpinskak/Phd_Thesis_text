% Chapter IV
\makeatletter
\def\input@path{{../}}
\makeatother
\documentclass[../main.tex]{subfiles}
\begin{document}
\chapter{Cloud voids - interpretation and explanation} % Chapter title

\label{ch:holes} % For referencing the chapter elsewhere, use \autoref{ch:name} 

This chapter presents experimental, theoretical and numerical results concerning cloud voids. Cloud voids are a phenomenon that was registered only once and was published for the first time in \citet{Karpinska2019} by the author of this thesis and others, including the experimental group. \ref{ch4s1} presents the experimental results in details. \ref{ch4s2} proposes the necessary conditions for cloud void formation, assuming the model exploited in \autoref{ch:single} in polydisperse particle case and verifies them in cloud-like conditions.

%----------------------------------------------------------------------------------------
\section{Cloud voids experiment results}
\label{ch4s1}
Cloud voids observations were performed at UFS on Zugspitze slopes in August 2011. Experimental methods used were described in Sec.\autoref{ch2s4}. This section presents the measurement results.\\
First, 30-minute long records of turbulence and droplet properties corresponding to the camera acquisition series in two measurement days were chosen for analysis. Droplet size raw measurements are presented in Fig. \ref{fig:ch4_1} and the corresponding statistics in Table \ref{tab:ch4_1}. Both cloud droplets, as well as drizzle drops were captured. On August 29th the droplet number concentration was visibly larger. Unfortunately the device deficiency did not allow for reliable measurement of the droplet concentration on August 29th. Next, the probability distribution of the droplet size has been calculated and presented in Fig.\ref{fig:ch4_2}. There are clear differences between the distributions measured on both days: the first one is much wider, the tail reach larger values and on average the droplets are about twice as big.

\begin{figure}[h]
\centering
\noindent\includegraphics[width=30pc]{gfx/Hist_counts_raw.png}
\caption{Histograms of droplet size counts measured with a PDI probe at the UFS on 27th and 29th of August 2011.}
\label{fig:ch4_1}
\end{figure}

\begin{figure}[h]
\centering
\noindent\includegraphics[width=30pc]{gfx/PDFs_log.png}
\caption{Droplet size probability distributions calculated for the data obtained with a PDI probe at the UFS on 27th and 29th of August 2011.}
\label{fig:ch4_2}
\end{figure}

High-resolution measurements of small-scale turbulence during cloud void events were conducted. Applying the methods described in Sec. \autoref{subs:atmosmeas}, mean energy dissipation rates and Kolmogorov scales were determined. Droplet and turbulence measured properties together with derived paremeters are summarized in Table \ref{tab:ch4_1}. Values of dimensionless parameters were calculated with the use of mean radius and Kolmogorov timescale. There is about one order of magnitude difference in $St$ between two cases, but the Froude numbers are comparable.

\begin{table}
\small
\tabcolsep=0.2cm
\caption{Properties of turbulence and cloud droplets during 30-minutes long observation periods. Values of dimensionless parameters are calculated with the use of mean radius.}
\centering
\begin{tabular}{|l|c|c|}
\hline
  & August 27th & August 29th\\
\hline
 Energy dissipation rate $\epsilon$ [cm$^2$/s$^3$] & 550 & 700 \\
\hline
 Kolmogorov length scale $\eta$ [mm] & 0.50  & 0.47\\
\hline
Komogorov timescale $\tau_{\eta}$ [ms] & 17 & 15\\
\hline
Droplet radius $R$ [$\mu$m] & 12.9 $\pm$ 4.8 & 6.4 $\pm$ 1.5\\
\hline
 Stokes number $St$ & 0.126  & 0.035\\
\hline
 Sedimentation parameter $S_v$ & 0.676  & 0.172\\
 \hline
 Froude number $Fr$ & 0.186  & 0.203\\
 \hline
 Number density $n$ [cm$^{-3}$] & 56 $\pm$ 47 & no data\\
 \hline
%\multicolumn{2}{l}{$^{a}$Footnote text here.}
\end{tabular}
\label{tab:ch4_1}
\end{table}

Multiple cloud images and movies were collected by laser-sheet imaging technique on the measurement days. In general two kinds of events in which droplet spatial distribution is visibly inhomogeneous were distinguished. The first kind is characterized by an irregular interface separating clear-air and cloudy-air volumes and/or cloudy volumes of visibly different properties over a wide range of spatial scales (panel b) in Fig. \ref{fig:ch4_3}). Inhomogeneities of the second kind, present within the cloudy volumes, were called cloud voids in ``Swiss cheese" clouds. Cloud voids were small (a few centimetres scale), the interface was usually blurry (see panels a) and c) in Fig. \ref{fig:ch4_3}) and the shapes of clear-air regions were often close to round or elliptic (see magnified voids in Fig. \ref{fig:ch4_4}). It is important to point out that the more intuitive expression "cloud holes" with regards to the second kind inhomogeneities is avoided on purpose because it is commonly used referring to the cloud-free regions occurring in stratocumulus decks, as described for example in \citet{Gerber_2005}.

\begin{figure}[h]
\centering
\noindent\includegraphics[width=35pc]{gfx/cloud_inhomog.png}
\caption{Examples of cloud voids observed at the UFS station with various camera-laser configurations. Images taken on 27 August (panel a) were chosen to estimate cloud void sizes. The ones recorded on 29 August evening (panel b) show the difference between inhomogeneities produced by cloud voids and those resulting from the mixing with clear air at the cloud edge. Other images from 29 August (panel c) suggest that the voids can be quite frequent in the sample volume. Bright spots and lines are due to presence of larger precipitation particles. 10~cm long segment is shown to represent spatial scale assumed in the void size calculation. For more details, see the movies attached in the supplementary materials.}
\label{fig:ch4_3}
\end{figure}

Inhomogeneities of the first kind are argued to be created in the process of cloud -- clear-air mixing (e.g. \cite{Warhaft_2000}). In contrast, in some series of images and movies, the shape of the recorded tracks of cloud droplets suggest the following cloud void origin hypothesis: they result from interactions between inertial, heavy cloud droplets and small-scale vortices present in a turbulent cloud. Comparison of the two described cases becomes straightforward when conducted on the basis of the enclosed movies \citep{database}. In the movie "ms01" between 13~s and 22~s there are two cloud void appearances. Motion of the void in the homogeneous cloud field resembles motion of a worm. Movie "ms02" presents cloudy and clear air mixing at the cloud edges. \\

\begin{figure}[h]
\centering
\noindent\includegraphics[width=35pc]{gfx/closeups.png}
\caption{Example close-ups of variously shaped cloud voids observed at the UFS station with different camera-laser configurations. 5~cm long section is placed in each image to represent spatial scale assumed in the void size calculation.}
\label{fig:ch4_4}
\end{figure}

There were a few series of cloud void images collected with various laser-camera settings on the two experimental days. The best quality series, made in the morning of the 27th, was chosen for void size analysis. For the series of 17 photos selected for analysis, there were four in which voids were not clear enough to be accounted for. In the remaining 13 photos 27 voids were identified. Each one's size was manually determined. In the case of a round void, the diameter was taken as the size; in a case of flattened or ellipsoidal void, the maximal chord was taken. The typical void diameter was estimated to be 3.5$\pm$1~cm; the maximal, 12$\pm$4~cm; the minimal, 1$\pm$0.5~cm. Images from the analysed series from the morning of August 27th showing examples of objects identified as voids are presented in the panel a) of Fig. \ref{fig:ch4_3}. Voids captured on the 29th of August were not analysed due to the large uncertainty resulting from the unknown geometry of the camera-laser set-up. The general experimental observation was that the voids were smaller then those on August 27th. Definitive experimental verification of the cloud void origin is not possible on the basis of collected data only; however, in next sections, I argue that void creation could be explained by interaction of inertial droplets present inside vortex tube.

%------------------------------------------------
\section{Cloud void creation conditions}
\label{ch4s2}
Lets assume that cloud voids are caused by the presence of a long-lasting vortex that appear numerously in turbulence structure. Is the model of single particle motion in a vortex explored in previous chapter (\autoref{ch:single}) able to produce a "void effect" when applied to cloud conditions? If yes, what are the necessary conditions? How they translate to model parameters? The goal of this section is to verify the hypothesis that the model is able to create a void effect, assuming cloud conditions, for some parameter sets, and that there are not any voids for other parameter cases.\\
In order to verify the hypothesis, a few steps are taken. Firstly, the void effect is defined in the language of particle trajectories. Secondly, instead of costly sweep through parameters domain in numerical simulations, one can use the analyticall results for single particle to try to approximate right parameter subdomain for void formation in polydisperse collection of droplets. Finally, for a few chosen parameter sets the designed vortex simulations are run and the spatial distribution of particles is presented to verify the void/no void cases.\\
Step one is void definition. Lets assume that we have a collection of significantly polydisperse cloud droplets, distributed uniformly in the air, where the appearance of a Burgers vortex of a certain size, circulation and gravity alignment, creates a void of a few centimetres size, lasting a few seconds. Here the void is defined as an inhomogeneity in droplet field, a region almost devoid of droplets. It has nearly cylindrical shape, in the cross section it takes a form close to a circle or an ellipsoid.
Judging by particle motion scenarios shown in the previous chapter, such a void is created in the model, when polydisperse particle collective behaviour is governed by the following rules. 
(11) the majority of droplets' trajectories are determined by limit cycle attraction. In the same time the attraction by a single stable equilibrium point far from the axis is not considerable, so the trajectories do not cross the void.
(22) the radius of curvature of the limit cycle of most of the droplets is large enough to make the region devoid of droplets distinguishable from mean spatial distribution fluctuations and should be comparable to the radius  of observed voids. 
(33) the time needed to form the void is shorter then exit time for most of the particles.
These conditions are inspected separately in the subsections and are converted to the model parameters conditions.

\subsection{Polydisperse droplet trajectories}
\label{ssec:poly}

%Obtaining a general mathematically strict condition for creation of an arbitrary sized void in arbitrary polydisperse collection of droplets would be too detailed and too complicated to be profitable for the interpretation of crude experimental results. 
Condition \ref{(11)} for the motion of the polydisperse collection of droplets in the 2D space are met as soon as most of the drops realize motion scenarios referred to in Fig. \autoref{fig:ch3_9} as (1b),(2b),(3b), and possibly only a small part of the largest droplets of scenario (5). Scenarios (1b),(2b),(3b) are realized under the condition that the particles have one unstable focus $r^+_0$ near the axis. As was proved in \autoref{sec:ch3s2}, this is in approximation equivalent the two inequalities: $r^+_0<r_s$ and $A<A_{max}(R,\delta,\theta)$. So these inequalities narrow the void-producing parameter domain down.In the last paragraphs of \autoref{ch3s2ss2} it was concluded, that the relation between $A_{max}$ and system parameters is intricate. In general it decreases with $\delta$ and $\theta$ in nonlinerar way, but the dependence on $R$ is not monotonic. Here I present an approach that can be more easily applied and interpreted in void formation problem.\\
If a vortex is described by $(A, \delta, \theta)$, I show below that the interval of particle radii that satisfy the two inequalities is bounded. Lets denote this interval by $[R_<,R_>]$.
The inequality $A<A_{max}(R,\delta,\theta)$ when $(A, \delta, \theta)$ are set, leads to the condition that $R  \in [R_1,R_2]$ . When $\theta \neq 0 $, then $R_1$,$R_2$ values can be obtained only numerically, just as it was described in the last paragraphs of \autoref{ch3s2ss2}.
The second inequality, $r^+_0<r_s$, might or might not narrow the interval coming from the first inequality further down. The second inequality is equivalent to $R<R_s$, where $R_s$ is defined as the radius of droplet that have its unstable focus at the distance $r_s$ from the vortex axis: $r^+_0(R_s,A,\delta,\theta)=r_s$, and can be calculated numerically as well. Combining the two inequalities together, one can see that $R_<=R_1$ and:
\begin{itemize}
\item if $R_s<R_2$, then  $R_>=R_s$,
\item if $R_s>R_2$, then $R_>=R_2$.
\end{itemize}
Figure \ref{fig:ch4_5} presents results of numerical calculations of $R_1$, $R_2$, $R_s$ in the vortex parameters domain, specific for cloud-like conditions (first, second and third row respectively). Colour scale was selected to be common to all three variables. Alignment angle $\theta$ cases are arbitrary, but smaller then $\pi/4$, for the clarity of the figures. For the larger the angle, the narrower the $A$ and $\delta$ domain in which limit cycle is a solution. The fourth row shows the difference $R_2-R_s$. $R_1$ decreases when strain parameter $A$ decreases (circulation increases) or vortex size $\delta$ decreases. On the contrary to $R_1$, $R_2$ and $R_s$ increase when strain parameter $A$ decreases (circulation increases) or vortex size $\delta$ decreases. The last row of subfigures reveals, that the sign of expression $R_2-R_s$  depends on the vortex parameters and confirms that the value of $R_>$ should be calculated twofold.

\begin{figure}[h]
\centering
\noindent\includegraphics[width=35pc]{gfx/R1R2Rs.png}
\caption{Approximated size range of particles $[R_1,R_2]$ that has a limit cycle solution, represented by color scale, with respect to vortex parameters (first and second row), $R_s$ (third row), as defined in the text body and the interval between $R_2$ and $R_s$ (fourth row), obtained numerically. Colorscale is common for $R_1$, $R_2$ and $R_s$, separate for $R_2-R_s$. Three alignment angle values selected are represented in columns. Vortex parameters domain corresponds to cloud-like conditions.}
\label{fig:ch4_5}
\end{figure}

Finally, \ref{fig:ch4_6} presents the interval of particle radii in which a droplet shows limit cycle behaviour in the case of vortex parameters, $(A, \delta, \theta)$, set. Values of $R_<$  reach from around 29~$\mu m$ for $\theta=\pi/16$ and around 19~$\mu m$ for $\theta=\pi/4$ to well below 1~$\mu m$, so they are surely relevant for cloud droplets. $R_>$ has opposite $A$ dependence and it reaches from around 17~$\mu m$ for all three $theta$ to particle sizes that are far out of cloud droplets bounds. What is interesting, is the interval magnitude $\Delta R$. The minimal values of those presented in the figures are $min(\Delta R)=27, 14, 4 \ \mu m$ for $\theta=\pi/16, \pi/8, \pi/4$ respectively. These values are of the same order as the width of droplet size ranges found in clouds (see \autoref{tab:ch2_01}). What is more, one can see that $R_<$ decreases when strain parameter $A$ decreases (circulation increases) or vortex size $\delta$ decreases. On the contrary, $R_2$ and $R_s$, so $R_>$ as well, increase when strain parameter $A$ decreases (circulation increases) or vortex size $\delta$ decreases.

\begin{figure}[h]
\centering
\noindent\includegraphics[width=35pc]{gfx/RlRrDeltaR.png}
\caption{Numerical calculation of approximated size range of particles $[R_<,R_>]$ (first and second row) that follow limit cycle behaviour, as well as interval between them $\Delta R$ (third row) in relation with vortex parameters. Colorscale is common for $R_>$ and $\Delta R$. Three alignment angle values selected are represented in columns. Vortex parameters domain corresponds to cloud-like conditions.}
\label{fig:ch4_6}
\end{figure}

Altogether some general clues on polydisperse particle trajectories are drawn:
\begin{enumerate}
\item There is a threshold (minimal) value of circulation needed for limit cycle appearance (corresponding to $A_{max}$). It increases with inclination angle $\theta$ and vortex size $\delta$.
\item The greater the circulation the smaller particles present limit cycle behaviour.
\item The range of particles having unstable points near the axis increases with increasing circulation and decreasing inclination angle and vortex size $\delta$.
\end{enumerate}

\noindent Building up on these results it may be concluded that theoretical void formation conditions can be fulfilled in cloud-like conditions when it comes to 2D space. According to the model it should be harder to observe voids in vortices of larger $\delta$. When it comes to the inclination, the larger the angle, the smaller the particles thrown out of vortex center. On the other hand, increasing inclination angle decreases the range of particles circling around the void. However, larger droplets motion resemble then sedimentation through the vortex and altogether increasing inclination angle might facilitate void formation. The last conclusion is that it should be more difficult to observe voids the larger particle size range is.\\

\subsection{Void size estimation}
\label{ssec:par}
The curvature of droplet trajectories should be large enough for a void to be noticeable. In order to estimate the curvature radius we perform the following reasoning. In the face of lack of analytical limit cycle solution, droplet trajectory curvature radius can be approximated by the periodic orbit radius, which was explored in detail in \autoref{ch3s2ss1sss1}. For this reason, stable periodic orbit radius numerical calculation results are presented in a novel form in Fig.\ref{fig:ch4_30} for various representative vortex parameters. Every color represents one of droplet sizes: $R=3,13,23$~$\mu m$ chosen to be within the experimental range for August 27th (see Table \ref{tab:ch4_1}). The figure illustrates contour plots for selected orbit radii (0.5~cm ,2~cm, 5~cm), corresponding to cloud void sizes observed. Overlapping (blue on a top, then pink and green) coloured surfaces match subspaces of stable periodic orbit existence. 

Figure \ref{fig:ch4_30} shows the results of $r_{orb}$ calculation in a different view. 
\begin{figure*}
\centering
\noindent \includegraphics[width=30pc]{gfx/void_radii_R_3_13_23.png}
\caption{Contour plot of stable periodic orbit radius for droplets of radii $R=3,13,23 \ \mu m$ for cloud-like parameter ranges of $\delta$ and $A$. Overlapping (blue on a top, then pink and green) coloured surfaces match parameter domains in which stable periodic 2D orbit exist for droplet radius given by its colour. Dashed lines are contour plots for $r_{orb}$ equal to 0.5~cm ,2~cm, 5~cm. Black points represent simulation parameters sets (filled later, referred to in ???).}
\label{fig:ch4_7}
\end{figure*}

Figure \ref{fig:ch4_7} further narrows down the void-forming parameter domain.

\subsection{Timescales of motion}
Timescales of motion found for a single particle can be interpreted in the case of polydisperse collection of particles. Lets take the arbitrary range of sizes $[R_<,R_>]$ and find the relations between their timescales. Firstly, motion along vortex axis is governed by $\tau_z$ quantity, which approximately does not depend on particle size. Difference in total exit times results from the difference of equilibrium point position $z_b \propto R^2$ and according to Eq.\ref{ch3:eq14}:
\begin{equation}
\frac{\tau_{ex >}}{\tau_{ex <}}=\frac{log(L(Z^*_>,z_{0 >}^*)}{L(Z^*_<,z_{0 <}^*)}.
\label{ch4:eq1}
\end{equation}
When it comes to circular orbit timescale, there is:
\begin{equation}
\frac{\tau_{orb >}}{\tau_{orb <}}=\frac{R_>}{R_<},
\label{ch4:eq2}
\end{equation}
so it increases proportionally with droplet size. Finally the relations for docking times are implicit. Figure \autoref{fig:ch3_4b} - tu opisac, ze tam gdzie dazy do nieskonczonosci czas dokowania, to wiadomo, ze szybciej wypadna przez $tau_z$ mniejsze. A takie regiony sa dwa. Moze ten wykres przerobic na delta i A, zeby byl porownywalny z tymi powyzej?
\section{Summary}
3D void formation due to the presence of Burgers vortex in droplets field is fully possible for cloud-like conditions.
%------------------------------------------------

\end{document}